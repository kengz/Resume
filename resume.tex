%%%%%%%%%%%%%%%%%%%%%%%%%%%%%%%%%%%%%%%%%
% Medium Length Professional CV
% LaTeX Template
% Version 2.0 (8/5/13)
%
% This template has been downloaded from:
% http://www.LaTeXTemplates.com
%
% Original author:
% Trey Hunner (http://www.treyhunner.com/)
%
% Important note:
% This template requires the resume.cls file to be in the same directory as the
% .tex file. The resume.cls file provides the resume style used for structuring the
% document.
%
%%%%%%%%%%%%%%%%%%%%%%%%%%%%%%%%%%%%%%%%%

%----------------------------------------------------------------------------------------
%	PACKAGES AND OTHER DOCUMENT CONFIGURATIONS
%----------------------------------------------------------------------------------------

\documentclass{resume} % Use the custom resume.cls style

\usepackage[left=0.75in,top=0.6in,right=0.75in,bottom=0.6in]{geometry} % Document margins
\usepackage{hyperref}
\usepackage{multicol}
\usepackage[usenames,dvipsnames]{xcolor}
% \usepackage{natbib}

\name{Wah Loon Keng} % Your name

\address{Box 7277, Lafayette College, 111 Quad Drive, Easton, PA 18042
, USA} % Your address
\address{\tt{
\hfill (484) 541-5913 
\hfill \href{mailto:kengw@lafayette.edu}{kengw@lafayette.edu} 
\hfill \href{https://github.com/kengz}{https://github.com/kengz} \hfill}} % Your phone number and email

% \address{ {\em An undergraduate theorist who loves solving real world problems in math, computer science and physics. }}



\begin{document}

%----------------------------------------------------------------------------------------
%	EDUCATION SECTION
%----------------------------------------------------------------------------------------
\begin{rSection}{Education}

{\bf Lafayette College, Pennsylvania} \hfill expected May 2016\\ 
B.S. in Mathematics, Minor in Computer Science \hfill Overall GPA: 3.88
\end{rSection}

%----------------------------------------------------------------------------------------
%	RESEARCHES
%----------------------------------------------------------------------------------------

\begin{rSection}{RESEARCHES}

\begin{rSubsection}{Lafayette College EXCEL Program}{Summer 2015}{Computational Geometry with Dr. Ge Xia.}{}
\item Vertex Cover, a signature NP-hard problem. {\em In progress.}
% lorem test \cite{yao13}.
\end{rSubsection}

%------------------------------------------------

\begin{rSubsection}{Perimeter Institute for Theoretical Physics Summer Student}{Summer 2014}{Quantum Foundations with Dr. Matthew Pusey and Dr. Tobiaz Fritz.}{}
\item Correlations in causal structure. Application of Quantum Computation and Information to the foundations of quantum mechanics. \\
{\em Notes:} {\tt \href{https://github.com/kengz/Quantum-Foundations-Correlations/blob/master/Keng%20Correlations%20in%20C3.pdf}{\textcolor{Cerulean}{Correlations in C3}}} and {\tt \href{https://github.com/kengz/Quantum-Foundations-Correlations/blob/master/Keng%20blockcode.pdf}{\textcolor{Cerulean}{BlockCode and Bundled Form}}, W.L. Keng}.
% test \cite{yao13}.
\end{rSubsection}

%------------------------------------------------

\begin{rSubsection}{Lafayette College EXCEL Program}{Summer \& Fall 2013}{Computational Geometry with Dr. Ge Xia.}{}
\item Delaunay Triangulation, graphs, spanner problems. Proved that the Yao-5 graph, useful in wireless networks, is a spanner, i.e. shortest distance always exists.\\
{\em Published:} {\tt \href{http://arxiv.org/abs/1307.5829}{\textcolor{Cerulean}{New and Improved Spanning Ratios for Yao Graphs}}, Barba et.al.}
% {\em GitHub:} {\tt \href{https://github.com/kengz/Yao-Graph-Research}{Yao-Graph-Research.}}
% lorem test \cite{yao13}.
\end{rSubsection}

%------------------------------------------------

\end{rSection}


%----------------------------------------------------------------------------------------
%	PROJECTS
%----------------------------------------------------------------------------------------

\begin{rSection}{Projects \href{https://github.com/kengz}{(on Github)}}

{\setlength{\parskip}{1.8pt}

\ \\
{\tt \href{https://github.com/kengz/lomath}{\textcolor{Cerulean}{lomath}}} - Data analytics, math module for NodeJS inspired by Lodash/Underscore.

{\tt \href{https://github.com/kengz/loML}{\textcolor{Cerulean}{loML}}} - A machine learning module for NodeJS, powered by lomath. 

{\tt \href{https://github.com/kengz/jarvis}{\textcolor{Cerulean}{jarvis}}} - A powerful butler bot for startups/team projects; cloud-deployable and cross-platform.

{\tt \href{https://github.com/kengz/telegram-bot-bootstrap}{\textcolor{Cerulean}{telegram-bot-bootstrap}}} - A bootstrap for Telegram bot with full API support, cloud-deployable.

{\tt \href{https://github.com/kengz/dokker}{\textcolor{Cerulean}{dokker}}} - Automated JS code documentation generator.

{\tt \href{https://github.com/kengz/Risk-game}{\textcolor{Cerulean}{Risk-game}}} - Simple AI to play the game Risk, with statistical analysis.

{\tt \href{https://github.com/kengz/Machines}{\textcolor{Cerulean}{Machines}}} - Implementation of Turing Machines, PDA, NFA, DFA, with a theoretical thesis on the link between their memories and powers.
}
\end{rSection}


%----------------------------------------------------------------------------------------
%	WORK EXPERIENCE
%----------------------------------------------------------------------------------------

\begin{rSection}{EMPLOYMENT}

{\setlength{\parskip}{1.8pt}
\ \\
{\sl Data Analysis, Automation, HTML Email Research, Fulcrum Tech, Inc} \hfill  2014 - Present

{\sl Graphic Designer and Proctor, Lafayette College Foreign Languages Dept.} \hfill  2014 - Present

{\sl Physics Supplemental Instructor/Grader, Lafayette College} \hfill  2012 - Spring 2014

}
\end{rSection}


%----------------------------------------------------------------------------------------
%	LANGUAGES
%----------------------------------------------------------------------------------------

\begin{rSection}{LANGUAGES}

\begin{tabular}{ @{} >{\bfseries}l @{\hspace{6ex}} l }
Computer 	& Proficient: {\tt Java, C++, NodeJS, HTML/CSS/Sass, Mathematica } \\
  			& Basic: {\tt R, Python, Matlab } \\
Spoken 		& Fluent: English, Chinese, Malay, Cantonese, Hokkien \\
\end{tabular}

\end{rSection}





\newpage


%----------------------------------------------------------------------------------------
%	WORK EXPERIENCE
%----------------------------------------------------------------------------------------

\begin{rSection}{ENTREPRENEURSHIP AND LEADERSHIP}

{\setlength{\parskip}{1.8pt}
\

\begin{rSubsection}{GLOBAL\_HACKERS}{}{Co-creator}{}
\item A tight network of social entrepreneurs, engineers, business development consultants, healthcare advocates, and researchers, collaborating on startups together, connected by a common goal to ameliorate global social issues through technology and innovation.
% lorem test \cite{yao13}.
\end{rSubsection}

%------------------------------------------------

\begin{rSubsection}{\href{http://splory.my}{Splory by Splore Inc.}}{}{Co-founder, Creative Director}{}
\item A Malaysian startup - a \href{http://splory.my}{\textcolor{Cerulean}{social hangout application}} that streamlines event-scheduling with friends, with a built-in network of local businesses for recommendations.
% lorem test \cite{yao13}.
\end{rSubsection}

}
\end{rSection}

%----------------------------------------------------------------------------------------
%	COURSES
%----------------------------------------------------------------------------------------

\begin{rSection}{Courses}

%Section content\ldots

\setlength{\columnsep}{1cm}
\begin{multicols}{2}

\textbf{Freshman}\\
\textsc{Math 263} 	 \	Calculus III (A)\\
\textsc{Math 264} 	 \	Differential Equations (A)\\
\textsc{Math 312} 	 \	Partial Differential Equations (A)\\
\textsc{Phys 151} \  \	Accelerated Physics (A)\\
\textsc{Phys 218} \  \	Oscillatory \& Wave Phenomena (A)

\textbf{Sophomore}\\
\textsc{Math 290} 	 \	Transition of Theoretical Math (A)\\
\textsc{Math 300} 	 \	Vector Spaces (A)\\
\textsc{Math 356} 	 \	Real Analysis I (B+)\\
\textsc{Phys 342} \	 \	Electromagnetic Fields (A)\\
\textsc{Phys 351} \	 \	Quantum Theory (A)\\
\textsc{Phys 327} \	 \	Advanced Classical Mechanics (A)\\
\textsc{Phys 338} \	 \	Advanced Physics Lab (A)

\textbf{Junior}\\
\textsc{CS 150} \ \ \ \  \	Data Structures and Algorithms (A)\\
\textsc{CS 205} \ \ \ \	 \	Software Engineering (A)\\
\textsc{CS 303} \ \ \ \	 \	Theory of Computation (A)\\
\textsc{Math 351} 		 \	Abstract Algebra (A-)\\
\textsc{Math 358} 		 \	Topology (B)\\
\textsc{Math 391} 		 \	Adv. Multivariable Calculus (A)\\
\textsc{Math 335} 		 \	Probability (A)\\

\textbf{Anticipated} (Fall 2015)\\
\textsc{CS 202} \ \ \ \	 \	Advanced Algorithms\\
\textsc{CS 420} \ \ \ \	 \	Artificial Intelligence\\
\textsc{Math 375} 		 Applied Fixed \& Mixed Effect Models\\


\end{multicols}

\end{rSection}


%----------------------------------------------------------------------------------------
%	AWARDS
%----------------------------------------------------------------------------------------

\begin{rSection}{AWARDS}
{\setlength{\parskip}{1.8pt}

\ \\
{\sl Lafayette College Benjamin F. Barge Mathematical Prize} \hfill 2014

{\bf Second, Third }{\sl - Lafayette College Barge Math Competitions } \hfill 2013 - 2014

{\bf Second }{\sl - LVAIC Regional College Math Competition} \hfill  2012

{\bf Dean's List }{\sl - Lafayette College} \hfill 2012 - 2014

{\bf Gold, Top 5 team }{\sl - National Physics Competition, Malaysia} \hfill 2010


{\bf Gold }{\sl - ICAS International Math Competitions, University of New South Wales} \hfill  2009}
\end{rSection}



%----------------------------------------------------------------------------------------
%	EXAMPLE SECTION
%----------------------------------------------------------------------------------------

%\begin{rSection}{Section Name}

%Section content\ldots

%\end{rSection}

%----------------------------------------------------------------------------------------

% \bibliographystyle{plain}
% name without .bib:
% \bibliography{research}

\end{document}
