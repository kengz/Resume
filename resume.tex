%%%%%%%%%%%%%%%%%%%%%%%%%%%%%%%%%%%%%%%%%
% Medium Length Professional CV
% LaTeX Template
% Version 2.0 (8/5/13)
%
% This template has been downloaded from:
% http://www.LaTeXTemplates.com
%
% Original author:
% Trey Hunner (http://www.treyhunner.com/)
%
% Important note:
% This template requires the resume.cls file to be in the same directory as the
% .tex file. The resume.cls file provides the resume style used for structuring the
% document.
%
%%%%%%%%%%%%%%%%%%%%%%%%%%%%%%%%%%%%%%%%%

%----------------------------------------------------------------------------------------
%	PACKAGES AND OTHER DOCUMENT CONFIGURATIONS
%----------------------------------------------------------------------------------------

\documentclass{resume} % Use the custom resume.cls style

\usepackage[left=0.75in,top=0.6in,right=0.75in,bottom=0.6in]{geometry} % Document margins
\usepackage{hyperref}
% \usepackage{natbib}

\name{Wah Loon Keng} % Your name

\address{Box 7277, Lafayette College, 111 Quad Drive, Easton, PA 18042
} % Your address
\address{\tt{
\hfill (484) 542-3520 
\hfill \href{mailto:kengw@lafayette.edu}{kengw@lafayette.edu} 
\hfill \href{https://github.com/kengz}{https://github.com/kengz} \hfill}} % Your phone number and email

\address{ {\em An undergraduate theorist who loves solving real world problems in math, computer science and physics. }}



\begin{document}

%----------------------------------------------------------------------------------------
%	EDUCATION SECTION
%----------------------------------------------------------------------------------------
\begin{rSection}{Education}

{\bf Lafayette College, Pennsylvania} \hfill expected May 2016\\ 
B.S. in Mathematics, Minor in Computer Science \hfill Overall GPA: 3.86\\
{\em An undergraduate theorist who loves solving real world problems in math, computer science and physics. }

\end{rSection}


%----------------------------------------------------------------------------------------
%	AWARDS
%----------------------------------------------------------------------------------------

\begin{rSection}{AWARDS}

{\setlength{\parskip}{1.8pt}

\ \\
{\sl Lafayette College Benjamin F. Barge Mathematical Prize} \hfill 2014

{\bf First, Second, Third }{\sl - Lafayette College Barge Math Competitions } \hfill 2013 - 2014

{\bf Second }{\sl - LVAIC Regional College Math Competition} \hfill  2012

{\bf Dean's List }{\sl - Lafayette College} \hfill 2012 - 2014

{\bf Gold, Top 5 team }{\sl - National Physics Competition, Malaysia} \hfill 2010


{\bf Gold }{\sl - ICAS International Math Competitions, University of New South Wales} \hfill  2009}
\end{rSection}


%----------------------------------------------------------------------------------------
%	RESEARCHES
%----------------------------------------------------------------------------------------

\begin{rSection}{RESEARCHES}

\begin{rSubsection}{Fulcrum Tech, Inc}{Fall 2014 - Present}{Email Standard Research}{}
\item Developed HTML email template that renders consistently with responsiveness across all email clients.\\
{\em GitHub (private):} {\tt \href{https://github.com/kengz}{HTML-Email-with-Sass, html-email-generator}}.
\end{rSubsection}

%------------------------------------------------

\begin{rSubsection}{Perimeter Institutes for Theoretical Physics Summer Student}{Summer 2014}{Quantum Foundations with Dr. Matthew Pusey and Dr. Tobiaz Fritz.}{}
\item Correlations in the C3 causal structure. Used Quantum Computation and Information theory to study the foundations and differences between quantum and classical physics. \\
{\em Unpublished:} {\tt \href{https://github.com/kengz/Quantum-Foundations-Correlations/blob/master/Keng%20Correlations%20in%20C3.pdf}{Correlations in C3}} and {\tt \href{https://github.com/kengz/Quantum-Foundations-Correlations/blob/master/Keng%20blockcode.pdf}{BlockCode and Bundled Form}, W.L. Keng}.
% test \cite{yao13}.
\end{rSubsection}

%------------------------------------------------

\begin{rSubsection}{Lafayette College EXCEL Program}{Summer \& Fall 2013}{Computational Geometry with Dr. Ge Xia.}{}
\item Delaunay Triangulation, graphs, spanner problems. Proved that the Yao-5 graph, useful in wireless networks, is a spanner, i.e. short distance always exists.\\
{\em Published:} {\tt \href{http://arxiv.org/abs/1307.5829}{New and Improved Spanning Ratios for Yao Graphs}, Barba et.al.}\\
{\em GitHub:} {\tt \href{https://github.com/kengz/Yao-Graph-Research}{Yao-Graph-Research.}}
% lorem test \cite{yao13}.
\end{rSubsection}

%------------------------------------------------

\end{rSection}


%----------------------------------------------------------------------------------------
%	WORK EXPERIENCE
%----------------------------------------------------------------------------------------

\begin{rSection}{WORK EXPERIENCE}

{\setlength{\parskip}{1.8pt}
\ \\
{\sl Email Researcher and Coder, Fulcrum Tech, Inc} \hfill  2014 - Present

{\sl Graphic Designer and Proctor, Lafayette College Foreign Languages Dept.} \hfill  2014 - Present

{\sl Physics Student Grader, Lafayette College} \hfill  2012 - Present

{\sl Physics Supplemental Instructor, Lafayette College} \hfill  Spring 2014

}
\end{rSection}


%----------------------------------------------------------------------------------------
%	LANGUAGES
%----------------------------------------------------------------------------------------

\begin{rSection}{LANGUAGES}

\begin{tabular}{ @{} >{\bfseries}l @{\hspace{6ex}} l }
Computer 	& Proficient: $\verb!Java, C++, HTML/CSS/Sass, Mathematica, LaTeX!$ \\
  			& Elementary: $\verb!C, Matlab!$ \\
Spoken 		& Fluent: English, Chinese, Malay, Cantonese, Hokkien \\
\end{tabular}

\end{rSection}

%----------------------------------------------------------------------------------------
%	EXAMPLE SECTION
%----------------------------------------------------------------------------------------

%\begin{rSection}{Section Name}

%Section content\ldots

%\end{rSection}

%----------------------------------------------------------------------------------------

% \bibliographystyle{plain}
% name without .bib:
% \bibliography{research}

\end{document}
